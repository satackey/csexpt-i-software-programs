% !TEX root = ../main.tex
\begin{document}

\section{作成したプログラムの説明}
まず、``初期値: ''を標準出力へ表示した上で、
標準入力からの読み込みを行う。

指示より、10個の擬似乱数を作成、表示するため、10回の繰り返し処理を行い、終了する。

繰り返し処理の内容を以下に示す。

まず、乱数の元となる数を二乗する。
その数から、以下の式を用いて、2桁目から6桁目までの数を算出した。
元にする数を$n^2$とする
\begin{align*}
    n^2 \div 100 \bmod 10000
\end{align*}

算出した値を、元となる数へ代入した上で、
値を標準入力に表示した。

\end{document}
