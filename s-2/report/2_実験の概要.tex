% !TEX root = ../main.tex
\begin{document}

\section{実験の概要}
\subsection{平方採中法について}
  平方採中法とは、1940年代にノイマンによって提案された、擬似乱数を生成するための古典的手法である。
  例えば、求めたい乱数の桁数を最大で4桁とし、乱数の初期値を6184とすると、次のような手順で順次求められる。

  \begin{align*}
    6174^2 = 38118276 → 1182 \\
    1182^2 = 01397124 → 3971 \\
    3971^2 = 15768841 → 7688
  \end{align*}

  このような、与えられた値を二乗し、その間の特定の桁を乱数として採用し、
  その値を次の初期値として擬似乱数を算出するのが、平方採中法である。

\subsection{作成するプログラムの仕様}
  \subsubsection{入力}
    入力された$n$を初期値として、平方採中法により求められた10個の擬似乱数、
    ここでは、最大で4桁の擬似乱数を求めるものとする。


\end{document}
