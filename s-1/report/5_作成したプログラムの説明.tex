% !TEX root = ../main.tex
\begin{document}

\section{作成したプログラムの説明}
まず、入力を整数値で受け取る。

指示された仕様である、$n$の値が$0$または$6174$に達するまでという条件は、
$n$が0でも6174でも無い間は、指示された操作を行うということである。

したがって、回数を数える変数$i$を0で初期化した上で、
その条件を継続条件として、次の処理を繰り返し行う。

並べ変えてできる、最大値、最小値を計算するために、各桁の値を0〜9として取り出す。
この処理を行うとき、i番目の値を取り出すために、以下の計算式を参考に実装した。
\begin{align*}
    (i番目の値) = n \div 10^{i-1} \bmod 10
\end{align*}
取り出した値を4個の長さの配列として扱う。


次に、最大値、最小値の計算方法を説明する。

4個の数値の配列を、最大値の場合は降順、最小値の場合は昇順にソートする。
1,2,3,4などといった4個の数値の配列から、1234という1つの数値を算出するため、
以下の式を用いた上で、その和を計算することで、実装した。
4個の数値の配列を$digits$として、計算式を以下に示す。
\begin{align*}
    \sum_{i=0}^4 digits_i \times 10^{3-i}
\end{align*}

ここまでの処理で、nを並び替えてできる最大値、最小値の計算を行うことができる。
最大値、最小値の差を計算する。
ここで変数$i$をインクリメントし、繰り返しの処理はここで終了する。

繰り返しの継続条件を満たすかどうかを確認し、満たさなければ、ここで繰り返しを終了する。

最後に$i$を出力してプログラムを終了する。

\end{document}
