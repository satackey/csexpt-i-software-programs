% !TEX root = ../main.tex
\begin{document}

\section{ソフトウェアテストの結果について}
入力内容と期待される出力内容を表\ref{入力内容と期待される出力内容}に示す。

\begin{table}[H]
    \centering
    \caption{入力内容と期待される出力内容}
    \label{入力内容と期待される出力内容}
        \begin{tabular}{|c|c|} \hline
            入力内容 & 期待される出力内容 \\ \hline
            0001 & 5 \\
            2447 & 6 \\
            3883 & 5 \\
            7676 & 4 \\
            9592 & 4 \\
            9999 & 1 \\ \hline
        \end{tabular}
\end{table}

以下より、作成したプログラムが正しく動作することを確認した。

\$ ./s-1.out \\
正の整数(4桁)の入力: 9999 \\
1 \\
\\
\$ ./s-1.out \\ 
正の整数(4桁)の入力: 9552 \\ 
4 \\
\\
\$ ./s-1.out \\
正の整数(4桁)の入力: 2447 \\
6 \\
\\
\$ ./s-1.out \\
正の整数(4桁)の入力: 7676 \\
4 \\
\\
\$ ./s-1.out \\
正の整数(4桁)の入力: 3883 \\
5 \\
\\
\$ ./s-1.out \\
正の整数(4桁)の入力: 9999 \\
1 \\

\end{document}

