% !TEX root = ../main.tex
\begin{document}

\section{ソフトウェアテストの結果について}
入力内容と期待される出力内容、および下記の実行結果より実際の出力内容、期待される結果と実際の結果の比較を表\ref{入力内容と期待される出力内容}に示す。

\begin{table}[H]
    \centering
    \caption{ソフトウェアテストの結果および}
    \label{入力内容と期待される出力内容}
        \begin{tabular}{|c|c|c|c|} \hline
            入力内容 & 期待される出力内容 & 実際の出力内容 & 期待される結果と実際の結果の比較(成功かどうか) \\ \hline
            0001 & 5 & 5 & 成功 \\
            2447 & 6 & 6 & 成功 \\
            3883 & 5 & 5 & 成功 \\
            7676 & 4 & 4 & 成功 \\
            9592 & 4 & 4 & 成功 \\
            9999 & 1 & 1 & 成功 \\ \hline
        \end{tabular}
\end{table}
入力値が境界に近い場合での動作を確認するため、0001および9999をテストの入力に用いた。
また、通常の動作を確認するため、ランダムに数値を選び、2447から9592までの入力を用いた。

以下にプログラムの実行結果を示す。 \\
%
\$ ./s-1.out \\
正の整数(4桁)の入力: 0001 \\
5 \\
\\
\$ ./s-1.out \\
正の整数(4桁)の入力: 2447 \\
6 \\
\\
\$ ./s-1.out \\
正の整数(4桁)の入力: 3883 \\
5 \\
\\
\$ ./s-1.out \\
正の整数(4桁)の入力: 7676 \\
4 \\
\\
\$ ./s-1.out \\ 
正の整数(4桁)の入力: 9552 \\ 
4 \\
\\
\$ ./s-1.out \\
正の整数(4桁)の入力: 9999 \\
1 \\

表\ref{入力内容と期待される出力内容}の結果より、期待された出力を行っているから、プログラムリスト\ref{作成したプログラム}は正しい。

\end{document}
