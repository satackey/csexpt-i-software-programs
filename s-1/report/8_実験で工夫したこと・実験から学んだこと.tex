% !TEX root = ../main.tex
\begin{document}

\section{実験で工夫したこと・実験から学んだこと}
関数を定義することで、できる限りコードの再利用を可能にした。
最大値、最小値の計算のためにそれぞれ関数を用意するのではなく、
2つの違いは、ソート順の違い、つまりソート時の条件のみが異なる点に着目し、
引数で最大値、最小値計算の区別を行うことで、共通化を行い、コードを再利用することができた。

また、C言語の関数では配列を戻り値に使用することができなかったため、
配列の変数は最初の要素のポインタであることを利用して、
本来戻り値を代入する変数を関数呼び出し前に定義して、ポインタを引数に渡すことで、
擬似的に戻り値を受け取るように実装した。

\end{document}
