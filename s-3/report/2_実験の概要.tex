% !TEX root = ../main.tex
\begin{document}

\section{実験の概要}
\subsection{モンテカルロ法について}
  モンテカルロ法は、多数の乱数を使用することにより数値計算における近似解を求めることを目的とする。その有名な例が、円周率の近似値計算である。
  一辺の長さが1の正方形内に半径が1の中心角が90度のおうぎ形を書いた上で、n個の点をランダムにプロットしたとき、おうぎ形部分の内側に入った点の個数をPとする。

  この場合、
  \begin{align*}
    \dfrac{4P}{n}
  \end{align*}
  が,円周率の近似値となる.

\subsection{作成するプログラムの仕様}
  \subsubsection{入力}
    正の整数 $n$ を入力とする
  
  \subsubsection{出力}
    モンテカルロ法においてn個の点をプロットすることにより計算された円周率の近似値を出力とする。
    ただし、ここでは点の$x$、$y$座標(それぞれ0~1の範囲)を $(double)rand() / RAND\_MAX$ により計算する。

\end{document}
