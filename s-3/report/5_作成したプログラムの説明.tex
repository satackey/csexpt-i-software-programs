% !TEX root = ../main.tex
\begin{document}

\section{作成したプログラムの説明}
擬似乱数を用いてx,y座標を定め、その座標がおうぎ形の内側であるかどうかを返す関数を作成した。
整数の擬似乱数を生成し、生成されうる最大値で割って0〜1の範囲の実数の擬似乱数を2回算出し、x,yに代入する。
円の内側かを判断するために、原点から距離を計算する以下の式を用いた。
\begin{align*}
  \sqrt{x^2 + y^2}
\end{align*}
この計算結果が1以下であればおうぎ形の内側である。

実行の流れを以下に示す。
まず、``nを入力: ''を標準出力へ表示した上で、
標準入力からの読み込みを行う。
a
作成した関数の用いてn回のうちで円の内側に入った点の個数をカウントする。
指示より、$(カウントの数) \times 4 \div n$を計算、出力して終了する。

\end{document}
